\subsection{Leveraging Linguistic Information}
\label{sec:useTime}

% Do we have any experiments that do take the positions of terms in the
% sequence of videos into account? Such an experiment would nicely tell
% a story from naive TF/IDF to including pedagogy to including external
% wikipedia.

Considering all of the n-grams in the collection of lectures as
candidate keywords has the potential of adding significant noise. By
selecting only certain linguistic patterns for consideration as
phrases, it is possible to reduce the size of the candidate set, while
still covering most important phrases. We experimented with an
algorithm that first runs a part-of-speech tagger over the lecture
transcripts, and then selected only phrases that consist of an
arbitrary number of adjectives followed by one or more nouns. For
example, ``equality condition'' or ``XML data'' were both included in
the candidate set. We then ran TF-IDF over this reduced set. After
this process we proceeded as in Section~\ref{sec:tfidf}.
