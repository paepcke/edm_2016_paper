\section{Experiments}
\label{sec:exp}

% Introduce what we are trying to do: explore different methods for
% building the index, and comparing against gold. Foreshadow the
% methods: naive/pedagogy-inclusive/external-knowledge.

% List whatever is common to the three experiments: stop word removal,
% method used for stemming...

Our first experiment took a traditional approach, selecting words for
the index that appeared disproportionately often in certain lectures
(TF-IDF \cite{mann2008}). We then incorporated lexical information, by
only considering phrases that followed certain part-of-speech
patterns. Finally, we introduced external knowledge from Wikipedia
into an algorithm's indexing decisions. Note that none of the
algorithms included supervised learning, as we do not assume the
existence of a training set for all courses. The following subsections
introduce the algorithm (families) beyond the TF-IDF version.

%% We implemented several index term extraction algorithms and measured
%% how closely they agreed with the gold index derived from the work of
%% our human indexers. 

% Insert something about the metric once we have it really nailed down.

% For each experiment we applied the Porter stemming algorithm to each
% word in the document and each word in phrases destined for the index.
% Phrases therefore matched if all of the individual stemmed tokens
% matched. In experiments using n-grams as candidate phrases, stopwords
% were removed from the document before the n-grams were formed, using
% the stopword list of the SMART system \cite{salton1971smart}.

% In experiments using part-of-speech tagging, the Stanford Log-Linear POS tagger was used \cite{toutanova2003feature}.

% The following subsections introduce the algorithm (families) we
% applied to the lecture transcripts.
